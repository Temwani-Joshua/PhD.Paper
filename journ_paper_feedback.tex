\documentclass[]{article}

%opening
%\title{}
%\author{}

\begin{document}

%\maketitle

%\begin{abstract}

%\end{abstract}

\section{Comments from Prof D}
The paper continues to make a good impression on actually reading it, so well done! However, it will still need a lot of work of course. 
 
One overall question which jumped into my mind – how do these FEKO results differ from simply coding up the Fresnel reflection coefficients in a MATLAB code which works out the direct and reflected rays, assuming a point source? This is rather important, since doing the latter definitely isn’t novel.
My question was actually – can’t we do more sophisticated stuff now with FEKO and the MoM? That would really add to the novelty. 

TJ – further to this, what I’m suggesting we could possibly do with FEKO is to include something like a small hill and model it as Hardie did, using Geometrical Optics, or something metallic like a container and use MLFMM? This one absolutely CAN’T do without an MoM code.

\end{document}
